\documentclass{article}
\usepackage[utf8]{inputenc} %input 
\usepackage[spanish]{babel} %language
\usepackage{indentfirst} %first line of a par indented
\usepackage{listings} %inserting code
\usepackage{enumitem} %custom lists
\usepackage[a4paper, total={6in, 8in}]{geometry}
\usepackage{hyperref} %reference other places, URL, etc 
\usepackage[most]{tcolorbox} %Custom boxes
\usepackage{multicol} %mutiple columns on a single paper
\usepackage{lipsum} %text filler
\usepackage{fancyhdr} %custom paper layout
\usepackage{tikz} %custom figures
\usepackage{adjustbox} %Cajas bro
\renewcommand{\baselinestretch}{1.2}

\newcommand\ChangeRT[1]{\noalign{\hrule height #1}}
\hypersetup{
	colorlinks=true,
	citecolor=black,
	filecolor=black,
	linkcolor=black,
	urlcolor=black
}

\title{Práctica PS-ECI: Procesador Simple - Estructura de Computadores I}
\author{
	Seguí Vives, Mateu\\
	\textit{mateufassers6b@gmail.com}\\
	\texttt{20549548R}
	\and
	Palmer Pérez, Rubén\\
	\textit{ruben.palmer1@estudiant.uib.cat}\\
	\texttt{43474448D}
	\and
	\\Curso 2019-2020\\
	\texttt{Estructura de Computadores I - Grupo (\textbf{1})}
}
\date{\today}

\pagestyle{fancy}
\fancyhf{}
\rhead{Práctica PS-ECI: Procesador Simple}
\rfoot{\thepage}

\begin{document}
	\begin{titlepage}
	\centering
	\includegraphics[width=0.15\textwidth]{UIB.png}\par\vspace{1cm}
	{\scshape\LARGE Universitat de les Illes Balears \par}
	\vspace{1cm}
	{\scshape\Large Práctica final EC\par}
	\vspace{1.5cm}
	{\huge\bfseries Práctica PS-ECI: Procesador Simple\par}
	\vspace{2cm}
	{\large\begin{multicols}{2}
	Seguí Vives, Mateu\\
	\textit{mateufassers6b@gmail.com}\\
	\texttt{20549548R}
	\vfill\null
	\columnbreak
	Palmer Pérez, Rubén\\
	\textit{ruben.palmer1@estudiant.uib.cat}\\
	\texttt{43474448D}
	\end{multicols}\par}
	\vfill


% Bottom of the page
	{\large \texttt{Grupo 01} \\\today\par}
    \end{titlepage}
	\thispagestyle{empty}
	\newpage
	
	\thispagestyle{empty}
	\tableofcontents
	\newpage
	
	\section{Introduction}
	En esta práctica se nos presenta con el reto de emular una máquina llamada \textit{PS-ECI (Procesador Simple - Estructura de Computadores I)} en el \textbf{68K} usando los conocimientos que hemos ido adquiriendo a lo largo de nuestro curso.\\
	\\
	Las instrucciones y registros de la \textit{PS-ECI} son de 16bits. A lo largo de esta documentacion, a la hora de hablar sobre registros emulados, direcciones emuladas, etc, se usara el pronombre \textbf{e}.
	\subsection{Registros e Instrucciones}
	\noindent A continuacion se mostraran los registros que tiene esta máquina y sus respectivas funciones:
	\begin{itemize}
		\item \textbf{T0, T1 -} Propio de operaciones de tipo \textbf{ALU} e interfaz con a memoria principal.
		\item \textbf{R2, R3, R4, R5 -} De uso general ademas de usarse en algunas operaciones de tipo \textbf{ALU}.
		\item \textbf{B6, B7 -} Registros de direcciones usados en algunas instrucciones mediante el direccionamiento indirecto.
		\item \textbf{EIR -} Registro de instrucción.
		\item \textbf{EPC -} Contador de programa.
		\item \textbf{ESR -} Registro de estado que guarda los flags \textit{Zero}, \textit{Negative} y \textit{Carry} tal que: (00000000 00000ZNC).	
	\end{itemize}	
	
	\section{Descripción general}
	Debemos simular cada uno de los pasos que una maquina debe hacer a la hora de ejecutar cualquier instrucción. Será necesario, pues, emular la fase de fect, decodificación y la propia ejecución
	\subsubsection{\textit{fetch}}
	
	Empezando con la fase de \textit{fetch}, tenemos el contador de programa \textbf{EPC} que apunta a la siguiente instrucción. Sin embargo hay que tener en cuenta que la \textit{SP-ECI} trabaja con registros de 16bits mientras que la maquina del \textit{68K} trabaja con bytes. Para arreglar estre problema, \textbf{multiplicaremos por 2 nuestro EPC}. Una vez tenemos la direccion de la siguiente instruccion, podemos buscarla en la memoria emulada \textbf{EPROG} y la transferimos a \textbf{EIR}
	
	\newpage
	\subsubsection{\textit{decodificación}}
Para la fase de decodificación, cogemos la einstrucción contenida en \textbf{EIR} y la sometemos a una serie de \textbf{BTST} hasta que conseguimos identificar la instrucción a realizar .Una vez identificamos que instrucción es, devolvemos un valor que se usará más adelante para ejecutar dicha instrucción.
    
	\subsubsection{\textit{ejecución}}
	Según el valor que nos devuelva la fase de \textit{decodificación}, se ejecutara una de las siguientes instrucciones:
	\begin{itemize}
	    \item \textbf{0. EXIT: }Detiene la máquina
	    \item \textbf{1. COPY: }Carga el contenido de cualquier eregistro $<b>$ a otro eregistro $<c>$
	    \item \textbf{2. ADD: }Suma los contenidos de dos eregistros $<a>$ y $<b>$ y guarda el resultado en el eregistro $<c>$
	    \item \textbf{3. SUB: }Resta los contenidos de dos eregistros $<a>$ y $<b>$ y guarda el resultado en el eregistro $<c>$
	    \item \textbf{4. AND: }Realiza la \textit{AND} bit a bit lógica entre los eregistros $<b>$ y $<c>$ y el resultado es guardado en el eregistro $<c>$
	    \item \textbf{5. NOT: }Realiza la \textit{NOT} bit a bit lógica del eregistro $<c>$ y el resultado es guardado en si mísmo.
	    \item \textbf{6. STC: }Guarda una constante \textbf{k} con extensión de signo en un eregistro $<c>$
	    \item \textbf{7. LOA: }Guarda el contenido de una edirección \textbf{M} en un eregistro \textbf{T0} o \textbf{T1} (dependendiendo de una variable \textbf{i} codificada en la instrucción)  
	    \item \textbf{8. LOAI: }
	    Guarda el contenido del contenido del eregistro \textbf{B6} o \textbf{B7} (dependiendo de una variable \textbf{j} codificada en la instrucción) en el eregistro \textbf{T0}
	    \item \textbf{9. STO: }Guarda el contenido de un eregistro \textbf{T0} o \textbf{T1} (dependendiendo de una variable \textbf{i} codificada en la instrucción) en la edirección \textbf{M}
	    \item \textbf{10. STOI: } Guarda el contenido del eregistro \textbf{T0} en el contenido de \textbf{B6} o \textbf{B7} (dependiendo de una variable \textbf{j} codificada en la instrucción)
	    \item \textbf{11. BRI: } Branch incondicional dado por una edirección \textbf{M} 
	    \item \textbf{12. BRC: }Branch condicional dado por una edirección \textbf{M} y el eflag \textbf{C}
	    \item \textbf{13. BRZ: }Branch condicional dado por una edirección \textbf{M} y el eflag \textbf{Z}
	\end{itemize}
	\section{Rutina de decodificación}
	Para decodificar las instrucciones, usamos una subrutina de libreria \textbf{DECOD} que usa la instrucción \textbf{BTST} para mirar el valor del bit actual en orden del más significativo al menos significativo. Dependiendo del valor del bit, se hace el salto correspondiente a la siguiente etiqueta hasta llegar a la que coincide con el valor de la instrucción. Previamente, se reservará un espacio en la pila para que la subrutina nos devuelva el valor asociado de la instrucción y haremos un push del eregistro \textbf{EIR}. A continuación se presenta el arbol de decodificación
	\vspace{1cm}
	
\noindent\makebox[\textwidth]{
\begin{tikzpicture}[->, sibling distance = 1cm]
\node{INSTRUCCIÓN}
	child{node{0}
		child{node{00}
		    child{node{}
		        child{node{}
		            child{node{}
		                child{node{EXIT}}}}}
		    child[missing]
		    child[missing]}      
		child[missing]
		child[missing]
		child{node{01}
			child{node{010}
				child{node{0100}
					child{node{01000}
							child{node{COPY}}}
					child{node{01001}
							child{node{ADD}}}}
				child[missing]
				child{node{0101}
					child{node{01010}
							child{node{SUB}}}
					child{node{01011}
							child{node{AND}}}}}
			child[missing]
			child[missing]
			child{node{011}
				child{node{0110}
					child{node{01100}
							child{node{NOT}}}
					child{node{01101}
							child{node{STC}}}}}}}
	child[missing]
	child[missing]
	child[missing]
	child[missing]
	child[missing]
	child[missing]
	child[missing]
	child{node{1}
		child{node{10}
			child{node{100}
				child{node{1000}
					child{node{}
						child{node{LOA}}}}
				child{node{1001}
					child{node{}
						child{node{LOAI}}}}}
			child[missing]
			child{node{101}
				child{node{1010}
					child{node{}
						child{node{STO}}}}
				child{node{1011}
					child{node{}
						child{node{STOI}}}}}}
		child[missing]
		child[missing]
		child[missing]
		child{node{11}
			child{node{110}
				child{node{1100}
					child{node{}
						child{node{BRI}}}}
				child{node{1101}
					child{node{}
						child{node{BRC}}}}}
			child[missing]
			child{node{111}
				child{node{1110}
					child{node{}
						child{node{BRZ}}}}
				child[missing]}}}
\end{tikzpicture}
}\par

\section{Tabla de subrutinas}
\begin{center}
\begin{adjustbox}{center}
\begin{tabular}{ccccc}
\ChangeRT{1.25pt}
\textbf{Etiqueta}     & \textbf{Librería}     & \textbf{Entrada}      & \textbf{Funcionalidad}& \textbf{Salida}  \\ \ChangeRT{1.25pt}

\multicolumn{1}{c|}{FLAGSNC} & \multicolumn{1}{c|}{NO} & \multicolumn{1}{c|}{SR} & \multicolumn{1}{c|}{Obtención flags ( No modifica acarreo)} & D6  \\ \hline

\multicolumn{1}{c|}{FLAGS} & \multicolumn{1}{c|}{NO} & \multicolumn{1}{c|}{SR} & \multicolumn{1}{c|}{Obtención flags (modifica acarreo)} & D6  \\ \hline

\multicolumn{1}{c|}{GET\_A} & \multicolumn{1}{c|}{NO} & \multicolumn{1}{c|}{EIR} & \multicolumn{1}{c|}{Obtención parámetro A} & A0 \\ \hline

\multicolumn{1}{c|}{GET\_B} & \multicolumn{1}{c|}{NO} & \multicolumn{1}{c|}{EIR} & \multicolumn{1}{c|}{Obtención parámetro B} & A1 \\ \hline

\multicolumn{1}{c|}{GET\_C} & \multicolumn{1}{c|}{NO} & \multicolumn{1}{c|}{EIR} & \multicolumn{1}{c|}{Obtención parámetro C} & A2 \\ \hline

\multicolumn{1}{c|}{GET\_K} & \multicolumn{1}{c|}{NO} & \multicolumn{1}{c|}{EIR} & \multicolumn{1}{c|}{Obtención de la constante \textbf{k}} & D2 \\ \hline

\multicolumn{1}{c|}{GET\_M} & \multicolumn{1}{c|}{NO} & \multicolumn{1}{c|}{EIR} & \multicolumn{1}{c|}{Obtención de la constante \textbf{M}} & D2 \\ \hline

\multicolumn{1}{c|}{GET\_I} & \multicolumn{1}{c|}{NO} & \multicolumn{1}{c|}{EIR} & \multicolumn{1}{c|}{\begin{tabular}[c]{@{}c@{}}Obtención del registro indicado \\ por el parámetro I\\ $I = 0 \implies TO$ \\ $I = 1 \implies T1$ \end{tabular}} & A4 \\ \hline

\multicolumn{1}{c|}{GET\_J} & \multicolumn{1}{c|}{NO} & \multicolumn{1}{c|}{EIR} & \multicolumn{1}{c|}{\begin{tabular}[c]{@{}c@{}}Obtención del registro indicado \\ por el parámetro J\\ $J = 0 \implies B6$ \\ $J = 1 \implies B7$ \end{tabular}} & A4 \\ \hline

\multicolumn{1}{c|}{REGDEC} & \multicolumn{1}{c|}{NO} & \multicolumn{1}{c|}{D4} & \multicolumn{1}{c|}{\begin{tabular}[c]{@{}c@{}}Obtención del registro mencionado \\ en los parámetros A,B y C \end{tabular}} & A4 \\ \hline

\multicolumn{1}{c|}{DECOD} & \multicolumn{1}{c|}{SI} & \multicolumn{1}{c|}{\begin{tabular}[c]{@{}c@{}}SUBQ.W \#2,SP \\ MOVE.W EIR,-(SP)\end{tabular}} & \multicolumn{1}{c|}{\begin{tabular}[c]{@{}c@{}}Decodificacion de la \\instrucción en EIR \end{tabular}} & {\begin{tabular}[c]{@{}c@{}} ADDQ.W \#2,SP \\ MOVE.W (SP)+,D1\end{tabular}} \\ 
\end{tabular}
\end{adjustbox}
\end{center}
	\section{Tabla de registros del 68K}
    \begin{center}
\begin{tabular}{ccc}
\ChangeRT{1.25pt}

\textbf{Registros}    & \textbf{Función}      & \textbf{Utilización} \\ \ChangeRT{1.25pt}

\multicolumn{1}{c|}{A0} & \multicolumn{1}{c|}{\begin{tabular}[c]{@{}c@{}}Cálculo del PC \\ 
Dirección del eregistro \textless{}a\textgreater{} \end{tabular}} & \begin{tabular}[c]{@{}c@{}}Fase de Fetch \\ Como parámetro \textless{}a\textgreater{} \end{tabular}     \\ \hline

\multicolumn{1}{c|}{A1} & \multicolumn{1}{c|}{\begin{tabular}[c]{@{}c@{}}Cálculo de índice de instrucción \\ Dirección del eregistro \textless{}b\textgreater{} \end{tabular}} & \begin{tabular}[c]{@{}c@{}}Salto a la fase de ejecución\\Como parámetro \textless{}b\textgreater{}  \end{tabular}   \\ \hline

\multicolumn{1}{c|}{A2} & \multicolumn{1}{c|}{ Dirección del eregistro \textless{}c\textgreater{}} & Como parámetro \textless{}c\textgreater{}  \\ \hline

\multicolumn{1}{c|}{A3} & \multicolumn{1}{c|}{Edirección M} & Parámetro M   \\ \hline

\multicolumn{1}{c|}{A4} & \multicolumn{1}{c|}{Masking} & Cálculo de registros \\ \hline

\multicolumn{1}{c|}{D0} & \multicolumn{1}{c|}{Contenido del eregistro \textless{}a\textgreater{}} & Como parametro \textless{}a\textgreater{} \\ \hline

\multicolumn{1}{c|}{D1} & \multicolumn{1}{c|}{\begin{tabular}[c]{@{}c@{}}Indice del salto a instrucción \\ Contenido del eregistro \textless{}b\textgreater{} \end{tabular}} & \begin{tabular}[c]{@{}c@{}}Salto a la fase de ejecución \\ Como parámetro \textless{}b\textgreater \end{tabular} \\ \hline

\multicolumn{1}{c|}{D2} & \multicolumn{1}{c|}{\begin{tabular}[c]{@{}c@{}}Almacenar el parámetro K \\ Almacenar el parámetro M \end{tabular} } & Ejecución de instrucciones \\ \hline

\multicolumn{1}{c|}{D3} & \multicolumn{1}{c|}{Calculo de saltos} & BRC, BRZ \\ \hline

\multicolumn{1}{c|}{D4} & \multicolumn{1}{c|}{Masking} & GET\_A, GET\_B, GET\_C GET\_K, REGDEC  \\ \hline

\multicolumn{1}{c|}{D5} & \multicolumn{1}{c|}{Flag Z} &  \\ \cline{1-2}
\multicolumn{1}{c|}{D6} & \multicolumn{1}{c|}{Flag N} & Cálculo de Flags\\ \cline{1-2}
\multicolumn{1}{c|}{D7} & \multicolumn{1}{c|}{Flag C} &                  
\end{tabular}
\end{center}
\noindent En este apartado cabe comentar un par de puntos que creemos importantes. Aunque gran parte de las subrutinas se pudiesen hacer de libreria ( usando la pila ) creemos que la molestia de trabajar supera al beneficio de tener más registros para el usuario. De esta manera, hemos intenado usar la pila lo menos posible usando, por ejemplo, los registros \textbf{D5, D6 \& D7} únicamente para los flags. Haciendolo de esta manera hacemos que el programa sea mucho más simple de comprende y tenemos el efecto secundario de no tener que preocuparnos de saber si ese registro aya esta siendo usado por otra rutina .
\newpage
	
	\section{Conjunto de pruebas}
Para comprobar que nuestra emulación se ejecuta correctamente ibamos a hacer una prueba por cada instrucción que falta por usar, ya que tanto el programa de prueba como el programa principal compilan y funcionan como se espera. Sin embargo acabamos creado  un  código  que  utilizase  estas instrucciones no usadas para hacer el conjunto de pruebas más conciso.
\begin{center}
\begin{adjustbox}{center}
\begin{tabular}{cccc}
\ChangeRT{1.25pt}
\textbf{@PS-ECI}            & \textbf{Ensamblador}           & \textbf{Codificación}                    & \textbf{Hex} \\ \ChangeRT{1.25pt}
\multicolumn{1}{c|}{0:}  & \multicolumn{1}{c|}{LOA C,T0} & \multicolumn{1}{c|}{1000 0000 0000 1100} & 800C         \\ \hline
\multicolumn{1}{c|}{1:}  & \multicolumn{1}{c|}{LOA D,T1}       & \multicolumn{1}{c|}{1000 1000 0000 1101} & 880D        \\ \hline
\multicolumn{1}{c|}{2:}  & \multicolumn{1}{c|}{AND T0,T1,T0}  & \multicolumn{1}{c|}{0101 1000 0000 1000} & 5808         \\ \hline
\multicolumn{1}{c|}{3:}  & \multicolumn{1}{c|}{NOT T0} & \multicolumn{1}{c|}{0110 0000 0000 0000} & 6000         \\ \hline
\multicolumn{1}{c|}{4:}  & \multicolumn{1}{c|}{SUB T0,T0,T0}     & \multicolumn{1}{c|}{0101 0000 0000 0000} & 5000         \\ \hline
\multicolumn{1}{c|}{5:}  & \multicolumn{1}{c|}{BRC 7}       & \multicolumn{1}{c|}{1101 0000 0000 0111} & D007         \\ \hline
\multicolumn{1}{c|}{6:}  & \multicolumn{1}{c|}{EXIT}  & \multicolumn{1}{c|}{0000 0000 0000 0000} & 0000        \\ \hline
\multicolumn{1}{c|}{7:}  & \multicolumn{1}{c|}{STC \#4,T0}       & \multicolumn{1}{c|}{0110 1000 0010 0000} & 6820         \\ \hline
\multicolumn{1}{c|}{8:}  & \multicolumn{1}{c|}{ADD T0,T1,T1}       & \multicolumn{1}{c|}{0100 1000 0000 1001} & 4809         \\ \hline
\multicolumn{1}{c|}{9:}  & \multicolumn{1}{c|}{COPY T1,B6}       & \multicolumn{1}{c|}{0100 0000 0000 1110} & 400E         \\ \hline
\multicolumn{1}{c|}{A:}  & \multicolumn{1}{c|}{STOI (B6)}       &
\multicolumn{1}{c|}{1011 0000 0000 0000} & B000         \\ \hline
\multicolumn{1}{c|}{B:} & \multicolumn{1}{c|}{EXIT}         &\multicolumn{1}{c|}{0000 0000 0000 0000} & 0000         \\ \hline
\multicolumn{1}{c|}{C:} & \multicolumn{1}{c|}{000A}         & \multicolumn{1}{c|}{0000 0000 0000 1010} & 000A         \\ \hline
\multicolumn{1}{c|}{D:} & \multicolumn{1}{c|}{0002}         & \multicolumn{1}{c|}{0000 0000 0000 0002} & 0002         
\end{tabular}
\end{adjustbox}
\end{center}\par

\noindent Las primeras instrucciónes, cargan \#\$A y \#2 en \textbf{T0} y \textbf{T1}, respectivamente. Una vez hacemos la \textbf{AND} T0 deberia contenter un \#2Hex donde inmediatamente después lo negamos, metiendo en \textbf{T0} el valor \#\$FFFD. Al restar \textbf{T0} con \textbf{T0} nos da 0 poniendo los flags \textbf{Z} y \textbf{C} a 1, aprovecharemos el flag  \textbf{C} para hacer un salto condicional. La siguiente instrucción es un \textbf{STC} que pone en \textbf{T0} el valor \#4Hex. Justo después, sumamos \textbf{T0} con \textbf{T1} lo que nos devuelve un \#6Hex en \textbf{T1} y copiamos ese valor a \textbf{B6}. Finalmente, Hacemos un \textbf{STOI} con \textbf{B6} lo que nos guarda en la posición C de nuestro eprograma el contenido de \textbf{T0} (\#4Hex)\\

\noindent Este programa no tiene 'sentido' ya que solo queremos probar que las instrucciones que no se habian tratado antes funcionen. Tras su ejecución, en la dirección @\$100C debería haber un 4Hex\par


	\section{Conclusiones}
	\noindent Como conclusión podemos decir que esta práctica nos ha ayudado a terminar de asimilar conceptos previamente adquiridos en la asignatura, ya que a la hora de la prácitca se ve la idea con mayor claridad. Además, la práctica nos ha servido de ayuda para estudiar de cara a los exámenes.\\
	
	\noindent Como un plus, esta práctica nos ha permitido prácticar \LaTeX{} y algo de programación. La documentación como se ha mencionado previamente, esta hecha por completo en \LaTeX{}, ademas se programó un pequeño projecto en java para poder hacer programas de prueba de la máquina. El código fuente del programa y del documento de \LaTeX{} se puede encontrar \textbf{\href{https://github.com/Zygmut/PRAFIN20}{AQUI}}
	
	\section{Código fuente}
	\addtolength{\oddsidemargin}{-.175in}
	\addtolength{\evensidemargin}{-.875in}
	\addtolength{\textwidth}{1.75in}

\begin{verbatim}
*-----------------------------------------------------------
* Title      : PRAFIN20
* Written by : Mateu Segui Vives, Ruben Palmer Perez
* Date       : 20/05/2020
* Description: Emulador de la PS-ECI
*-----------------------------------------------------------
    ORG $1000
EPROG: DC.W $8810,$400A,$E00D,$688E,$9000,$4003,$E00D,$6804
       DC.W $6FFD,$48A4,$495B,$E00D,$C009,$4020,$A012,$0000
       DC.W $0004,$0003,$0000
EIR:   DC.W 0 ;eregistro de instruccion
EPC:   DC.W 0 ;econtador de programa
ET0:   DC.W 0 ;eregistro T0
ET1:   DC.W 0 ;eregistro T1
ER2:   DC.W 0 ;eregistro R2
ER3:   DC.W 0 ;eregistro R3
ER4:   DC.W 0 ;eregistro R4
ER5:   DC.W 0 ;eregistro R5
EB6:   DC.W 0 ;eregistro B6
EB7:   DC.W 0 ;eregistro B7
ESR:   DC.W 0 ;eregistro de estado (00000000 00000ZNC)


START:
    CLR.W EPC

FETCH:
    ;--- IFETCH: INICIO FETCH
        ;*** En esta seccion debeis introducir el codigo necesario para cargar
        ;*** en el EIR la siguiente instruccion a ejecutar, indicada por el EPC
	    ;*** y dejar listo el EPC para que apunte a la siguiente instruccion
	
        MOVE.W EPC, A0
        ADD.W A0,A0     
        MOVE.W EPROG(A0),EIR
        ADDQ.W #1,EPC

    ;--- FFETCH: FIN FETCH
    
    ;--- IBRDECOD: INICIO SALTO A DECOD
        ;*** En esta seccion debeis preparar la pila para llamar a la subrutina
        ;*** DECOD, llamar a la subrutina, y vaciar la pila correctamente,
        ;*** almacenando el resultado de la decodificacion en D1

        SUBQ.W #2, SP 
        MOVE.W EIR,-(SP) 
        JSR DECOD
        ADDQ.W #2,SP
	MOVE.W (SP)+,D1

    ;--- FBRDECOD: FIN SALTO A DECOD
    
    ;--- IBREXEC: INICIO SALTO A FASE DE EJECUCION
        ;*** Esta seccion se usa para saltar a la fase de ejecucion
        ;*** NO HACE FALTA MODIFICARLA

    MULU #6,D1
    MOVEA.L D1,A1
    JMP JMPLIST(A1)
JMPLIST:
    JMP EEXIT 
    JMP ECOPY 
    JMP EADD 
    JMP ESUB 
    JMP EAND 
    JMP ENOT 
    JMP ESTC 
    JMP ELOA 
    JMP ELOAI 
    JMP ESTO 
    JMP ESTOI 
    JMP EBRI 
    JMP EBRC 
    JMP EBRZ    

    ;--- FBREXEC: FIN SALTO A FASE DE EJECUCION
    
    ;--- IEXEC: INICIO EJECUCION
        ;*** En esta seccion debeis implementar la ejecucion de cada einstr.

EEXIT:
	SIMHALT 
	
ECOPY:
	JSR GET_B 
	JSR GET_C 
	MOVE.W (A1),D1
	MOVE.W D1, (A2)
	JSR FLAGSNC
	BRA FETCH

EADD:
	JSR GET_A 
	JSR GET_B 
	JSR GET_C 
	MOVE.W (A0),D0
	MOVE.W (A1),D1
	ADD.W D0,D1
	JSR FLAGS
	MOVE.W D1,(A2)
	BRA FETCH

ESUB:
	JSR GET_A 
	JSR GET_B 
	JSR GET_C 
	MOVE.W (A0),D0 
	MOVE.W (A1),D1 
	NEG.W D1
	ADD.W D0,D1
	JSR FLAGS
	MOVE.W D1,(A2)
	BRA FETCH

EAND:
	JSR GET_A 
	JSR GET_B 
	JSR GET_C 
	MOVE.W (A0),D0
	MOVE.W (A1),D1
	AND.W D0,D1
	JSR FLAGSNC
	MOVE.W D1,(A2)
	BRA FETCH

ENOT:
	JSR GET_C 
	MOVE.W (A2),D0
	NOT.W D0
	JSR FLAGSNC
	MOVE.W D0,(A2)
	BRA FETCH
	
ESTC:
	JSR GET_K 
	JSR GET_C 
	MOVE.W D2,(A2)
	JSR FLAGSNC
	BRA FETCH

ELOA: 
	JSR GET_I 
	MOVE.W EIR, D2
	AND.W #$00FF, D2 ;Sacar M 
	MOVE.W D2,A3
	ADD.W A3,A3
	MOVE.W EPROG(A3),(A4)
	JSR FLAGSNC
	BRA FETCH

ELOAI:
	JSR GET_J 
	MOVE.W (A4),A4
	ADD.W A4,A4
	MOVE.W EPROG(A4),ET0
	JSR FLAGSNC
	BRA FETCH

ESTO:
	JSR GET_I
	MOVE.W EIR, D2
	AND.W #$00FF, D2
	MOVE.W D2,A3
	ADD.W A3,A3
	MOVE.W (A4),EPROG(A3)
	BRA FETCH

ESTOI:
	JSR GET_J 
	MOVE.W (A4),A4
	ADD.W A4,A4
	MOVE.W (ET0),EPROG(A4)
	BRA FETCH

EBRI:
	MOVE.W EIR, D2
	AND.W #$00FF, D2
	MOVE.W D2,EPC
	BRA FETCH

EBRC:
	MOVE.W ESR, D3
	BTST #0, D3 ;Test flag C
	BEQ FETCH 
	MOVE.W EIR, D2
	AND.W #$00FF, D2
	MOVE.W D2,EPC
	BRA FETCH

EBRZ:
	MOVE.W ESR, D3
	BTST #2, D3 ;Test flag C
	BEQ FETCH 
	MOVE.W EIR, D2
	AND.W #$00FF, D2
	MOVE.W D2,EPC
	BRA FETCH

    ;--- FEXEC: FIN EJECUCION

    ;--- ISUBR: INICIO SUBRUTINAS
        ;*** Aqui debeis incluir las subrutinas que necesite vuestra solucion
        ;*** SALVO DECOD, que va en la siguiente seccion

;Con respecto a los flags hemos decidido hacer la siguiente accion:
;Una subrutina para calcular los flags Z & N y otra únicamente para
;el flag C. De esta manera para actualizar todos los flags actuali-
;zaremos Z&N y luego actualizaremos Z 

FLAGSNC:
	MOVE.W SR, D5  
	MOVE.W D5, D6 
	AND.W #4,D5   ;Flag Z
	AND.W #8,D6   ;Flag N
	LSR.W #2,D6   
	OR.W D5,D6    
	MOVE.W D6,ESR 
	RTS

FLAGS:
	MOVE.W SR, D7
	JSR FLAGSNC
	AND.W #1,D7 ;Flag C
  	OR.W D7,D6  ;D6 <- ESR con Z & N
    	MOVE.W D6,ESR
    	RTS

GET_A: ;Pone en el registro A0 la dirección del operando a

	MOVE.W #$1C0, D4		
	AND.W EIR, D4
	LSR #6,D4
	JSR REGDEC
	MOVE.W A4,A0
	RTS	

GET_B: ;Pone en el registro A1 la dirección del operando b

	MOVE.W #$38, D4	
	AND.W EIR, D4
	LSR #3, D4
	JSR REGDEC
	MOVE.W A4,A1
	RTS

GET_C: ;Pone en el registro A2 la dirección del operando c

	MOVE.W #$7, D4	
	AND.W EIR, D4
	JSR REGDEC
	MOVE.W A4,A2
	RTS

GET_K: ;Pone en el registro D2 el valor k con extensión

    	MOVE.W #$7F8, D4	
	AND.W EIR, D4
	LSR #3, D4
	EXT.W D4
	MOVE.W D4,D2
	RTS

GET_M: ;Pone en el registro D2 el valor M

	MOVE.W EIR,D2
	AND.W #$00FF,D2
	RTS	

GET_I: ;Pone en el registro A4 la dirección de T1 o T0

	BTST #11, EIR ;Mirar si es T1 o T0
	BEQ RT0
RT1:	
	LEA.L ET1,A4
	RTS
RT0:
	LEA.L ET0,A4
	RTS

GET_J: ;Pone en el registro A4 la dirección de B6 o B7

	BTST #11, EIR ;Mirar si es B6 o B7
	BEQ RB6
RB7:	
	LEA.L EB7,A4
	RTS
RB6:
	LEA.L EB6,A4
	RTS
	
REGDEC: ;Decodifica los 3 bits menos significativos 
	;del registro D4 para determinar que 
	;registro emulado es

	BTST.L #2,D4
	BEQ R0
R1:
	BTST.L #1,D4
	BEQ R10
R11:
	BTST.L #0,D4
	BEQ R110
R111: ;EB7
	LEA.L EB7,A4
	RTS
R110: ;EB6
	LEA.L EB6,A4
	RTS	
R10:
	BTST.L #0,D4
	BEQ R100
R101: ;ER5
	LEA.L ER5,A4
	RTS
R100: ;ER4
	LEA.L ER4,A4
	RTS
R0:
	BTST.L #1,D4
	BEQ R00
R01:
	BTST.L #0,D4
	BEQ R010
R011: ;ER3
	LEA.L ER3,A4
	RTS		
R010: ;ER2
	LEA.L ER2,A4
	RTS
R00:
	BTST.L #0,D4
	BEQ R000
R001: ;ET1
	LEA.L ET1,A4
	RTS
R000: ;ET0
	LEA.L ET0,A4
	RTS

    ;--- FSUBR: FIN SUBRUTINAS

    ;--- IDECOD: INICIO DECOD
        ;*** Tras la etiqueta DECOD, debeis implementar la subrutina de 
        ;*** decodificacion, que debera ser de libreria, siguiendo la interfaz
        ;*** especificada en el enunciado

DECOD:
	BTST.B #7,4(SP)
	BEQ IN0
IN1:
	BTST.B #6,4(SP)
	BEQ IN10
IN11:
	BTST.B #5,4(SP)
	BEQ IN110
IN1110: ;BRZ
	MOVE.W #13,6(SP)		
	RTS
IN110:
	BTST.B #4,4(SP)
	BEQ IN1100
IN1101: ;BRC
	MOVE.W #12,6(SP)
	RTS
IN1100: ;BRI
	MOVE.W #11,6(SP)
	RTS
IN10:
	BTST.B #5,4(SP)
	BEQ IN100
IN101:
	BTST.B #4,4(SP)
	BEQ IN1010
IN1011: ;STOI
	MOVE.W #10,6(SP)
	RTS
IN1010: ;STO
	MOVE.W #9,6(SP)
	RTS
IN100:
	BTST.B #4,4(SP)
	BEQ IN1000
IN1001: ;LOAI
	MOVE.W #8,6(SP)
	RTS
IN1000: ;LOA
	MOVE.W #7,6(SP)
	RTS
IN0:
	BTST.B #6,4(SP)
	BEQ IN00
IN01:
	BTST.B #5,4(SP)
	BEQ IN010
IN0110: 
	BTST.B #3,4(SP)
	BEQ IN01100
IN01101: ;STC
	MOVE.W #6,6(SP)
	RTS
IN01100: ;NOT
	MOVE.W #5,6(SP)
	RTS
IN010:
	BTST.B #4,4(SP)
	BEQ IN0100
IN0101:
	BTST.B #3,4(SP)
	BEQ IN01010
IN01011: ;AND
	MOVE.W #4,6(SP)
	RTS
IN01010: ;SUB
	MOVE.W #3,6(SP)
	RTS
IN0100:
	BTST.B #3,4(SP)
	BEQ IN01000
IN01001: ;ADD
	MOVE.W #2,6(SP)
	RTS
IN01000: ;COPY
	MOVE.W #1,6(SP)
	RTS
IN00: ;EXIT
	MOVE.W #0,6(SP)
	RTS	

    ;--- FDECOD: FIN DECOD
	
    END START


\end{verbatim}
\end{document}
